\documentclass[11pt,a4paper]{article}
\usepackage[utf8]{inputenc}
\usepackage{amsmath}
\usepackage{amsfonts}
\usepackage{amssymb}
\usepackage{graphicx}
\usepackage{url}
\usepackage{enumitem}

\title{Semester project -- Pong Designer\\Theoretical report}
\author{Lomig Mégard}

\begin{document}
\maketitle

\section{Introduction}

\section{Type system}
The central point of Pong designer is its rule system. It requires the ability to reason about them, to deconstruct a rule to modify only a subset of it. This need forbids the use of raw Scala to handle rule content. The choice of an AST was clear since it is convenient to manipulate and modify in Scala. Each expression is thus evaluated by an interpreter at runtime. In order to be sure that an expression is correct, a system of dynamic type has been added to these trees. They are computed also at runtime. For the moment, the following types are handled. These types have a strong link to their corresponding Scala type. The translation in both directions is efficient and type safe.
\begin{itemize}[noitemsep,topsep=2pt,parsep=1pt,partopsep=1pt]
\item Integer
\item Floating point number
\item String
\item Boolean
\item Pair of floats
\end{itemize}

\paragraph*{}
The problem of typing was also raised by the user inputs. Indeed, XXX.

The game performances can suffer of the interpretation of rules. In a future, we could generate Scala code with the same behaviour as the trees. The compiled code would cohabits with its AST and thus benefits from both for the performance and the modularity.

\section{Categories of objects}

- Problem: collapsing multiples rules in one. Find a way to handle categories

- technical: implementation of categories.

\section{Physical engine}
The original goal if this project was to look at the physical engine used in the application and to find a way to improve it. That was especially needed since the implementation was suffering of tunnelling, namely when a physical object with a high velocity pass through another one. The physical engine was not the central point and it was a lost of time to maintain and debug. This is why we decided to integrate dedicated physical engine. 

I choose the project JBox2D\footnote{Website: \url{http://www.jbox2d.org/}} that exists for a long enough time to have good performances and a large community. For the moment we do not use all its features. Some could be useful in the future, for example the joins that permit to link two objects in multiple ways.

The architecture of JBox2D is based on the physical body. Each of them can have one or many shapes, defining its mass, center of mass, collision outline and other physical properties. To simplify the implementation, the principle of body has been kept in Pong Designer but some freedom have been removed. Particularly, each body has exactly one shape.

\section{Fixed time step}

- explain the different possibilities:
  
- draw the schema

- Fixed time step implementation (technical)



\section{Conclusion}



\end{document}